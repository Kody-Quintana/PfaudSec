\documentclass{article}
\usepackage{eso-pic}
\usepackage{datetime}
\usepackage{grffile}
\usepackage{lastpage}           %-to identify last page
\usepackage{xspace}             %-to force add a space
\usepackage[abspath]{currfile}  %-to find path to this file
\usepackage{xstring,ifthen,xifthen}     %-to manipulate strings
\usepackage{graphicx}           %-to add images
\usepackage{newunicodechar}
\usepackage{tabularx}
\usepackage{tikz}               %-to create Pfaudler logo
	\usetikzlibrary{shapes.geometric} %-for creating hexagon
	\usetikzlibrary{calc}
\usepackage{pgfplots} 
\pgfplotsset
  {
    compat                   = newest,
    xtick style={draw=none},
    %every tick/.append style = thin,
    width= .95 \textwidth,
    height= .80 \textheight
  }

\usepackage[linktoc=all,        %-for clickable links and pdf metadata
            colorlinks=true,
            linkcolor=black,
            urlcolor=black,
            citecolor=black,
            pdfauthor={Pfaudler},
            pdftitle={Quality Review},
            pdfsubject={},
            pdfkeywords={},
            pdfproducer={},
            pdfcreator={}]{hyperref}

\newdateformat{monthyeardate}{%
  \monthname[\THEMONTH], \THEYEAR}


\newcommand*{\fyyear}{\Roman{year}}

%-----Font configuration-----%
\usepackage{fontspec}
\setmainfont[
  Path          = ./,
  Ligatures     = TeX,
  UprightFont   = font/OTF/Pfaudler-Book.otf,
  BoldItalicFont= font/OTF/Pfaudler-BoldItalic.otf,
  BoldFont      = font/OTF/Pfaudler-Bold.otf,
  ItalicFont    = font/OTF/Pfaudler-BookItalic.otf
]{Pfaudler}
%----------------------------%


%\newfontfamily{\freeserif}{FreeSerif}

%-----Margin formatting-----%

%\usepackage[paperwidth=8.27in, paperheight=11.69in, landscape]{geometry}
%\usepackage[paperwidth=7.5in, paperheight=13.33in, landscape]{geometry}
\input{layout}
\geometry{top=0.9in, bottom=0.9in, left=0.9in, right=0.9in}
%---------------------------%



%-----Pfaudler colorscheme-----%
\definecolor{pfblue}{RGB}{52,134,199}
\definecolor{pfgrey}{RGB}{125,125,130}
%------------------------------%


%--------------Pfaudler logo (invoke with \pflogo)---------------%
%---(Must be contained in one paragraph for use in the header)---%

\newcommand\pflogo{%
\raisebox{-9pt}{%
\begin{tikzpicture}
\node [right,scale=1,inner sep=1pt,outer sep=0pt, text=pfblue]
at (0,0) {\fontsize{10.73}{5}\selectfont{%
%
%--Strapline--%
Defining the Standard}};
%
\node [right,scale=1,inner sep=0pt,outer sep=0pt, text=pfgrey]
at (0.01,0.70) {\addfontfeature{LetterSpace=0.0}\fontsize{30}{0}\selectfont{%
%
%--Manually kerned Main text--%
\textbf{%
P{\kern 0.0pt}%
f{\kern-1.0pt}%
a{\kern-1.1pt}%
u{\kern-1.1pt}%
d{\kern-1.1pt}%
l{\kern-1.1pt}%
e{\kern-1.1pt}%
r}%
}};
%
%--Pf hexagon--%
\path (4.70,1.32) node [regular polygon,
		regular polygon sides=6,
		rounded corners=1.5pt,
		inner sep=8.8pt,
		fill = pfblue](hexagon){};
%--Pf hexagon text--%
\node [text=white, scale=1.1] at (4.70,1.32) {\large{\textbf{Pf}}};
\end{tikzpicture}}}

%----------------------------------------------------------------%
\newcommand\pflogowhite{%
\raisebox{-9pt}{%
\begin{tikzpicture}
\node [right,scale=1,inner sep=1pt,outer sep=0pt, text=white]
at (0,0) {\fontsize{10.73}{5}\selectfont{%
%
%--Strapline--%
Defining the Standard}};
%
\node [right,scale=1,inner sep=0pt,outer sep=0pt, text=white]
at (0.01,0.70) {\addfontfeature{LetterSpace=0.0}\fontsize{30}{0}\selectfont{%
%
%--Manually kerned Main text--%
\textbf{%
P{\kern 0.0pt}%
f{\kern-1.0pt}%
a{\kern-1.1pt}%
u{\kern-1.1pt}%
d{\kern-1.1pt}%
l{\kern-1.1pt}%
e{\kern-1.1pt}%
r}%
}};
%
%--Pf hexagon--%
\path (4.70,1.32) node [regular polygon,
		regular polygon sides=6,
		rounded corners=1.5pt,
		inner sep=8.8pt,
		fill = white](hexagon){};
%--Pf hexagon text--%
\node [text=pfblue, scale=1.1] at (4.70,1.32) {\large{\textbf{Pf}}};
\end{tikzpicture}}}
%----------------------------------------------------------------%


%---------------------------------Header Formatting---------------------------------%
\usepackage{fancyhdr}
\pagestyle{fancy}

%--First page header style--%
\fancypagestyle{style1}{
\fancyhf{}
\fancyhead[L]{\pflogowhite}
\fancyhead[R]{
P.O. Box 23600, Rochester, NY 14692-3600\\
1000 West Ave. Rochester, NY 14611 USA\\
\begin{tabular}{@{}l r@{}}
Telephone:&\href{tel:5852351000}{1-585-235-1000}\\
Website:&\href{www.pfaudler.com}{www.Pfaudler.com}
\end{tabular}}
\fancyfoot[R]{}
\renewcommand{\headrulewidth}{0.4pt}}

%--Embedded documents overlay header--%
\fancypagestyle{style2}{
\fancyhf{}
\fancyhead[L]{\fontsize{25}{25}\selectfont\color{pfblue}\textbf{\the\year \xspace QA}}
\fancyfoot[R]{\thepage}
%\fancyfoot[R]{%
%\begin{tikzpicture}%
%\node [opacity=1] (0,0) {\textbf{Quality Review: }\thepage\xspace of \pageref{LastPage}};
%\end{tikzpicture}}
\renewcommand{\headrulewidth}{1pt}}
%-----------------------------------------------------------------------------------%
\newcolumntype{C}[1]{>{\centering\let\newline\\\arraybackslash\hspace{0pt}}m{#1}}


\newcommand{\pflogoart}{
\begin{tikzpicture}[remember picture, overlay]
\draw node[
	fill=pfblue,
	rounded corners=0.45in,
	regular polygon,
	regular polygon sides=6,
	minimum size=14in,
	text=white
](inner) at (current page.west) {};
\path  ($(inner.center)!0.44!(inner.east)$) node[text=white]{\fontsize{40}{40}\selectfont\textbf{
\begin{tabular}{C{6in}}
\input{name}\\
\monthyeardate\today
\end{tabular}}};
\end{tikzpicture}
}


%-----------Table of Contents Formatting-----------%
%----(compiler must be ran twice to update TOC)----%
\usepackage{titletoc}

%--section toc format--%
\titlecontents{section}
[1.5em]
{}
{\contentslabel{2.3em}}
{}
{\titlerule*[0.7pc]{.}\contentspage}

%--subsecion toc format--%
\titlecontents{subsection}
[4em]
{\hangindent1em}
{\contentslabel{2.3em}}
{}
{\titlerule*[0.7pc]{.}\contentspage}

%--Page number formatting--%
\makeatletter
\renewcommand{\contentspage}[1][\thecontentspage]{\hb@xt@\@pnumwidth{#1\hfil}\hspace*{-\@pnumwidth}}
\renewcommand{\@pnumwidth}{3em}
\makeatother

%--------------------------------------------------%



%%-----External job info (jobinfo.dat)-----%
%\usepackage{datatool}
%\DTLsetseparator{ = } %-Spaces must be included
%\DTLloaddb[noheader, keys={thekey,thevalue}]{jobinfo}{jobinfo.dat}
%\newcommand{\jobinfofill}[1]{\DTLfetch{jobinfo}{thekey}{#1}{thevalue}}
%%-----------------------------------------%



\setlength{\parskip}{0pt}
\setlength\headheight{62pt}



%-------------Page embedder-------------%
\usepackage{pdfpages}

%--newcommand for sections--%
\newcommand\addsection[1]{\addtocounter{section}{1}\phantomsection\addcontentsline{toc}{section}{{\arabic{section}.\xspace } {#1}}}

%--newcommand for subsections--%
\newcommand\addpage[1]{%
	\StrBefore[1]{#1}{.}[\temp]%-Display name must be stored in \temp because of expansion behavior in \includepdf
	\IfSubStr{\temp}{/}{\StrBehind{\temp}{/}[\temp]}{}%
	\StrSubstitute{\temp}{!}{ }[\temp]
	\includepdf[scale=0.95,pages=-,addtotoc={1,subsection,1,\temp,\temp},pagecommand=\phantomsection]{#1}}

%---------------------------------------%

%\newunicodechar{≥}{\makebox[.5em]{\freeserif≥}}


%%%%%                                                                                                       %%%%%
%%%%%  ____                   _             _____                                                      _    %%%%%
%%%%% |  _ \                 (_)           |  __ \                                                    | |   %%%%%
%%%%% | |_) |   ___    __ _   _   _ __     | |  | |   ___     ___   _   _   _ __ ___     ___   _ __   | |_  %%%%%
%%%%% |  _ <   / _ \  / _` | | | | '_ \    | |  | |  / _ \   / __| | | | | | '_ ` _ \   / _ \ | '_ \  | __| %%%%%
%%%%% | |_) | |  __/ | (_| | | | | | | |   | |__| | | (_) | | (__  | |_| | | | | | | | |  __/ | | | | | |_  %%%%%
%%%%% |____/   \___|  \__, | |_| |_| |_|   |_____/   \___/   \___|  \__,_| |_| |_| |_|  \___| |_| |_|  \__| %%%%%
%%%%%                  __/ |                                                                                %%%%%
%%%%%                 |___/                                                                                 %%%%%

\begin{document}
\AddToShipoutPictureBG*{\pflogoart}

\renewcommand{\headrulewidth}{0pt}
\setlength{\footskip}{20pt}
\pagestyle{style1}

\renewcommand{\headrulewidth}{0pt}
\newcolumntype{Y}{>{\centering\arraybackslash}X}


%%-----Table of Contents-----%
%\setcounter{tocdepth}{2}
%\setcounter{secnumdepth}{-2}
%\startcontents[section]
%\begin{center}
%\setlength{\parskip}{0.5em}
%\addsection{Table of Contents}
%\printcontents[section]{}{1}{}
%\setlength{\parskip}{1em}
%\end{center}
%%---------------------------%
\phantom{Placeholder for first page}
%-----Begin Embedded Documents-----%
\newpage %page style1 doesnt work without this newpage for some reason
%\setlength{\footskip}{20pt}
\pagestyle{style2}
\setlength\headheight{24pt}
\renewcommand{\headrulewidth}{0pt}
\include{graph}
%----------------------------------%


%\LARGE
%\begin{tikzpicture}% Line graph, use for data over time
%	\begin{axis}
%		[ymin=0,
%		title=\fontsize{25}{25}\selectfont\color{pfgrey}{Warranties Claims - Monthly},
%		tick label style={/pgf/number format/assume math mode},
%		every axis plot/.append style={ultra thick},
%		ymajorgrids,
%		bar width={0.06\textwidth},
%		legend style={
%			at={(0.5,-0.2)},
%			anchor=north,
%			legend columns=-1},
%		ylabel={},
%		symbolic x coords={
%			Jan,
%			Feb,
%			Mar,
%			Apr,
%			May,
%			Jun,
%			Jul,
%			Aug,
%			Sep,
%			Oct,
%			Nov,
%			Dec,
%			},
%		xtick=data,
%		nodes near coords,
%		nodes near coords align={vertical},
%		x tick label style={},]
%	\addplot [draw=pfblue,
%		nodes near coords={\pgfmathfloatifflags{\pgfplotspointmeta}{0}{}{\pgfmathprintnumber{\pgfplotspointmeta}}},
%		nodes near coords align={horizontal},
%		nodes near coords style={font=\Large,/pgf/number format/assume math mode}]
%		coordinates{
%		(Jan,3)
%		(Feb,8)
%		(Mar,0)
%		(Apr,2)
%		(May,0)
%		(Jun,0)
%		(Jul,0)
%		(Aug,0)
%		(Sep,0)
%		(Oct,0)
%		(Nov,0)
%		(Dec,0)
%		};
%	\end{axis}
%\end{tikzpicture}
%
%
%\begin{tikzpicture}% Bar graph, use for current month's values
%	\begin{axis}
%		[ybar,
%		ymin=0,
%		title=\fontsize{25}{25}\selectfont\color{pfgrey}{Warranties Claims - \monthyeardate\today},
%		tick label style={/pgf/number format/assume math mode},
%		every axis plot/.append style={ultra thick},
%		ymajorgrids,
%		bar width={0.06\textwidth},
%		legend style={
%			at={(0.5,-0.2)},
%			anchor=north,
%			legend columns=-1},
%		ylabel={},
%		symbolic x coords={
%			OE ≥ 4000,
%			Thing 2,
%			Thing 3,
%			Thing 4,
%			Thing 5,
%			Thing 6,
%			},
%		xtick=data,
%		nodes near coords,
%		nodes near coords align={vertical},
%		x tick label style={},]
%	\addplot [fill=pfblue,
%		draw=none,
%		nodes near coords={
%			\pgfmathfloatifflags
%			{\pgfplotspointmeta}{0}{}
%			{\pgfmathprintnumber{\pgfplotspointmeta}}},
%		nodes near coords align={south},
%		nodes near coords style={font=\Large,/pgf/number format/assume math mode},
%		every node near coord/.append style={xshift=0pt,yshift=-24pt,anchor=south,font=\color{white}\Large}]
%		coordinates{
%		(OE ≥ 4000,3)
%		(Thing 2,8)
%		(Thing 3,2)
%		(Thing 4,1)
%		(Thing 5,0)
%		(Thing 6,0)
%		};
%	\end{axis}
%% Pareto line
%	\begin{axis}
%		[ymin=0,
%		ymax=100,
%		tick label style={/pgf/number format/assume math mode},
%		every axis plot/.append style={ultra thick},
%		ytick style={draw=none},
%		%bar width={0.06\textwidth},
%		legend style={
%			at={(0.5,-0.2)},
%			anchor=north,
%			legend columns=-1},
%		ylabel={},
%		xtick=data,
%		xticklabels={,,},
%		yticklabel={\pgfmathprintnumber\tick\%},
%		xtick style={draw=none},
%		%nodes near coords,
%		%nodes near coords align={vertical},
%		yticklabel pos=right,
%		y tick label style={},
%		x tick label style={},]
%	\addplot [draw=orange,
%		nodes near coords={}
%%			\pgfmathfloatifflags
%%			{\pgfplotspointmeta}{0}{}
%%			{\pgfmathprintnumber{\pgfplotspointmeta}}},
%%		nodes near coords align={south},
%%		nodes near coords style={font=\Large,/pgf/number format/assume math mode},
%%		every node near coord/.append style={xshift=0pt,yshift=0pt,anchor=south,font=\color{orange}\Large}
%]
%		coordinates{
%		(0,3)
%		(1,30)
%		(2,40)
%		(3,50)
%		(4,60)
%		(5,100)
%		};
%	\end{axis}
%\end{tikzpicture}
\end{document}
